\documentclass[journal]{IEEEtran}

\usepackage{cite}

\usepackage{graphicx}  
%Please change to your image storage
\graphicspath{{../../Images/JPG/}}
\DeclareGraphicsExtensions{.pdf,.jpeg,.png}



\hyphenation{op-tical net-works semi-conduc-tor}


\begin{document}

\title{A Brief Group Introduction}

%++++++++++++++++++++++++
\author{Wenhao~Hong,~\IEEEmembership{Postgraduate,~XDU,}
        Feixiang~Liu,~\IEEEmembership{Postgraduate,~XDU,}
        Feng~Wang,~\IEEEmembership{Postgraduate,~XDU,}
        and~Renjun~Luo,~\IEEEmembership{Postgraduate,~XDU}% <-this % stops a space
}


%++++++++++++++++++++++++
% The paper headers
\markboth{Journal of \LaTeX\ Class Files,~Vol.~14, No.~8, September~2022}%
{Shell \MakeLowercase{\textit{et al.}}: Bare Demo of IEEEtran.cls for IEEE Journals}




% make the title area
\maketitle

% As a general rule, do not put math, special symbols or citations
% in the abstract or keywords.
\begin{abstract}
	This article is a simple introduction to the group members, which includes the photos, personality, group responsibility, and research interests of the group members.Through the reference of the document, and on the basis of the IEEE official template, we have completed this article.
\end{abstract}

% Note that keywords are not normally used for peerreview papers.
\begin{IEEEkeywords}
IEEE, Git Repository, Member Introduction
\end{IEEEkeywords}




\IEEEpeerreviewmaketitle



\section{Introduction}

%++++++++++++++++++++++++

\IEEEPARstart{A} clear and simple recommendation structure for scientific project repositories is essential for the entire scientific research project. Therefore, we built our own git repository based on the recommended repository structure proposed by Prof. Frery~\cite{2020A}.
At the same time, having a clear understanding of the internal characteristics of each team member is also helpful for subsequent learning and research.
Based on this, we submit a \LaTeX\ file in the repository with detailed profiles of each team member.

%+++++++++++++++++++++
\section{team member}

Wenhao Hong, from Jiangxi, has a lively personality and a wide range of interests. He likes to travel, listen to music and socialize. He does enjoy playing badminton, and often plays with friends during his free time. He loves to help others and once was a volunteer for the 14th Chinese National Games, who dressed up as a mascot to entertain others and to obtain happiness himself.

Liu Feixiang, from Handan City, Hebei Province, likes to read, sing, love to learn, sunny and optimistic, and be kind to people. During his undergraduate studies, he won the university-level scholarship for four consecutive years and participated in the innovation and entrepreneurship competition and won the provincial award. He is currently a graduate student of Professor Zhang Xiangrong.

Feng Wang, from Sichuan, likes playing badminton, loves sports, is outgoing and actively participates in various activities. He used to be a counselor assistant, deputy secretary of the third undergraduate branch of the AI Institute, and now is a postgraduate counselor of the AI Institute, with rich experience in student work. He studied actively and was successfully promoted as a postgraduate of Professor Xiangrong Zhang. 

Luo Renjun, a graduate student of the School of Artificial Intelligence, Class of 2022, studied in the School of Artificial Intelligence, Xi'an University of Electronic Science and Technology, and was guaranteed to continue his master's degree in the team of Professor Zhang Xiangrong in the School of Artificial Intelligence.

\bibliographystyle{unsrt}
%change it into your path
\bibliography{../Common/reference}


\begin{IEEEbiography}[{\includegraphics[width=1in,height=1.25in,clip,keepaspectratio]{wenhao.jpg}}]{Wenhao Hong}
(Postgraduate,~XDU) received his Bachelor of Engineering degree in Materials Science and Engineering from Xidian University in 2022, and won the second prize in the National English Competition for College Students in 2019. His current research interests are knowledge graph and object classification.ORCID Identifier: https://orcid.org/my-orcid?orcid=0000-0002-7394-5561
\end{IEEEbiography}


 \begin{IEEEbiography}[{\includegraphics[width=1in,height=1.25in,clip,keepaspectratio]{feixiang.jpg}}]{Feixiang Liu}
(Postgraduate, XDU) received a bachelor's degree from the School of Artificial Intelligence of Xidian University in 2018, won the second prize at the provincial level of the 2021 National College Students Mathematics Competition, and won several university-level scholarships. His research interests are in three-dimensional reconstruction.ORCID Identifier: https://orcid.org/my-orcid?orcid=0000-0003-3601-444X
\end{IEEEbiography}

  \begin{IEEEbiography}[{\includegraphics[width=1in,height=1.25in,clip,keepaspectratio]{wangfeng.jpg}}]{Feng Wang}
(Postgraduate, XDU) has won the excellent graduate students of Xidian University in 2020 and 2021, the first prize of Shaanxi Mathematical Contest in 2020, and the first prize of Shaanxi Mathematical Modeling Contest in 2020. Now the research direction is the detection application of knowledge graph.ORCID Identifier: https://orcid.org/my-orcid?orcid=0000-0002-3089-7934
 \end{IEEEbiography}

 \begin{IEEEbiography}[{\includegraphics[width=1in,height=1.25in,clip,keepaspectratio]{renjun.jpg}}]{Renjun Luo}
(Postgraduate,~XDU) won scholarships for four consecutive years during his undergraduate studies, and was awarded the title of Outstanding Student of Xidian University in 2021, the International First Prize in the 2020 US Student Mathematical Modelling Competition, and the First Prize in the 2020 Shaanxi Provincial Mathematical Modelling Competition. His research interests are remote sensing target tracking and detection.ORCID Identifier: https://orcid.org/my-orcid?orcid=0000-0001-5903-481X
\end{IEEEbiography}


\end{document}


